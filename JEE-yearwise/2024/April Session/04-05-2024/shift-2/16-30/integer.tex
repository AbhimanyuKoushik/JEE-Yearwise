\iffalse
\title{2024}
\author{EE24Btech11024 - G. Abhimanyu Koushik}
\section{integer}
\fi

\item Let the maximum and minimum values of $\brak{\sqrt{8x-x^2-12}-4}^2+\brak{x-7}^2$, $x\in\mathbb{R}$ be $M$ and $m$, respectively. Then $M^2-m^2$ is equal to \rule{1cm}{0.15mm}.

\hfill{\brak{\text{Apr 2024}}}

\item Let the point \brak{-1,\alpha,\beta} lie on the line of the shortest distance between the lines $\frac{x+2}{-3}=\frac{y-2}{4}=\frac{z-5}{2}$ and $\frac{x+2}{-1}=\frac{y+6}{2}=\frac{z-1}{0}$. Then $\brak{\alpha=\beta}^2$ is equal to \rule{1cm}{0.15mm}.

\hfill{\brak{\text{Apr 2024}}}

\item The number of real solutions of the equation $x\abs{x+5}+2\abs{x+7}-2=0$ is \rule{1cm}{0.15mm}.

\hfill{\brak{\text{Apr 2024}}}

\item Let $y=y\brak{x}$ be the solution to the differential equation $\frac{dy}{dx}+\frac{2x}{\brak{1+x^2}^2}y=xe^{\frac{1}{1+x^2}};y\brak{0}=0$. Then the area enclosed by the curve $f\brak{x}=y\brak{x}e^{-\frac{1}{1+x^2}}$ and the line $y-x=4$ is \rule{1cm}{0.15mm}.

\hfill{\brak{\text{Apr 2024}}}

\item Let a line perpendicular to the line $2x-y=10$ touch the parabola $y^2=4\brak{x-9}$ at the point $\vec{P}$. The distance of the point $\vec{P}$ from the centre of the circle $x^2+y^2-14x-8y+56=0$ is \rule{1cm}{0.15mm}.

\hfill{\brak{\text{Apr 2024}}}

\item The number of solutions of $\sin^{2}x+\brak{2+2x-x^2}\sin x - 3\brak{x-1}^2=0$, where $-\pi\leq x\leq \pi$, is \rule{1cm}{0.15mm}.

\hfill{\brak{\text{Apr 2024}}}

\item Let the mean and the standard deviation of a probability distribution\\
\begin{center}
   \begin{tabular}[12pt]{ |c|c|c|c|c|}
    \hline
    $X$ & $\alpha$ & $1$ & $0$ & $-3$\\
    \hline
    $P\brak{X}$ & $\frac{1}{3}$ & $K$ & $\frac{1}{6}$ & $\frac{1}{4}$\\
    \hline
    \end{tabular}
\end{center}
be $\mu$ and $\sigma$, respectively. Then $\sigma+\mu$ is equal to \rule{1cm}{0.15mm}.

\hfill{\brak{\text{Apr 2024}}}

\item If $1+\frac{\sqrt{3}-\sqrt{2}}{2\sqrt{3}}+\frac{5-2\sqrt{6}}{18}+\frac{9\sqrt{3}-11\sqrt{2}}{36\sqrt{3}}+\frac{49-20\sqrt{6}}{180}+\dots$ upto $\infty=2+\brak{\sqrt{\frac{b}{a}}+1}\log_e{\frac{a}{b}}$, where $a$ and $b$ are integers with $\gcd\brak{a,b}=1$, then $11a+18b$ is equal to \rule{1cm}{0.15mm}. 

\hfill{\brak{\text{Apr 2024}}}

\item If $f\brak{t}=\int_{0}^{\pi}\frac{2xdx}{1-\cos^{2}t\sin^{2}x}$, $0<t<\pi$, then the value of $\int_{0}^{\frac{\pi}{2}}\frac{\pi^{2}dt}{f\brak{t}}$ equals \rule{1cm}{0.15mm}.

\hfill{\brak{\text{Apr 2024}}}

\item Let $a>0$ be a root of the equation $2x^2+x-2=0$. If $\lim_{x\to\frac{1}{a}}\frac{16\brak{1-\cos\brak{2+x-2x^2}}}{\brak{1-ax}^2}=\alpha+\beta\sqrt{17}$, where $\alpha,\beta\in\mathbb{Z}$, then $\alpha+\beta$ is equal to \rule{1cm}{0.15mm}.

\hfill{\brak{\text{Apr 2024}}}
