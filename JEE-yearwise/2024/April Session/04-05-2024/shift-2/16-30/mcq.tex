\iffalse
\title{2024}
\author{EE24Btech11024}
\section{mcq-single}
\fi

\item Let $\vec{A}\brak{-1,1}$ and $\vec{B}\brak{2,3}$ be two points and $\vec{P}$ be a variable point above the line $AB$ such that the area of $\Delta PAB$ is 10. If the locus of $\vec{P}$ is $ax+by=15$, then $5a+2b$ is:

\hfill{\brak{\text{Apr 2024}}}
\begin{enumerate}
\begin{multicols}{4}
\item $6$
\item $4$
\item $-\frac{12}{5}$
\item $-\frac{6}{5}$
\end{multicols}
\end{enumerate}

\item Let $\alpha\beta\neq 0$ and $A=\myvec{\beta & \alpha & 3\\\alpha & \alpha & \beta\\-\beta & \alpha & 2\beta}$. If $B=\myvec{3\alpha & -9 & 3\alpha\\-\alpha & 7 & -2\alpha \\ -2\alpha & 5 & -2\beta}$ is the matrix of cofactor elements of $A$, then $\det\brak{AB}$ is equal to:

\hfill{\brak{\text{Apr 2024}}}
\begin{enumerate}
\begin{multicols}{4}
\item $216$
\item $343$
\item $64$
\item $125$
\end{multicols}
\end{enumerate}

\item The value of $m$, $n$ for which the system of linear equations \newline $x+y+z=4$, \newline $2x+5y+5z=17$, \newline $x+2y+mz=n$ \newline has infinitely many solutions satisfy the equation:

\hfill{\brak{\text{Apr 2024}}}
\begin{enumerate}
\begin{multicols}{2}
\item $m^2+n^2-m-n=46$	
\item $m^2+n^2+mn=68$
\item $m^2+n^2+m+n=64$
\item $m^2+n^2-mn=39$
\end{multicols}
\end{enumerate}

\item Let $ABCD$ and $AEFG$ be squares of side $4$ and $2$ units respectively. The point $\vec{E}$ is on the line segment $AB$ and the point $\vec{F}$ is on the diagonal $AC$. Then the radius $r$ of the circle passing through the point $\vec{F}$ and touching the line segments $BC$ and $CD$ satisfies:
\begin{enumerate}
\begin{multicols}{4}
\item $r=1$
\item $r^2-8r+8=0$
\item $2r^2-8r+7=0$
\item $2r^2-4r+1=0$
\end{multicols}
\end{enumerate}

\item Let $\vec{a}=2\hat{i}+5\hat{j}-\hat{k}$, $\vec{b}=2\hat{i}-2\hat{j}+2\hat{k}$ and $\vec{c}$ be three vectors such that $\brak{\vec{c}+\hat{i}}\times\brak{\vec{a}+\vec{b}+\hat{i}}=\vec{a}\times\brak{\vec{c}+\hat{i}}$. If $\vec{a}\cdot\vec{c}=-29$, then $\vec{c}\cdot\brak{-2\hat{i}+\hat{j}+\hat{k}}$ is equal to:

\hfill{\brak{\text{Apr 2024}}}
\begin{enumerate}
\begin{multicols}{4}
\item $15$
\item $12$
\item $5$
\item $10$
\end{multicols}
\end{enumerate}
